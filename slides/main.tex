% https://github.com/martinhelso/UiB


\documentclass[UKenglish]{beamer}

\usetheme{UiB}


\usepackage[utf8]{inputenx} % For æ, ø, å
\usepackage{csquotes}       % Quotation marks
\usepackage{microtype}      % Improved typography
\usepackage{amssymb}        % Mathematical symbols
\usepackage{mathtools}      % Mathematical symbols
\usepackage[absolute, overlay]{textpos} % Arbitrary placement
\setlength{\TPHorizModule}{\paperwidth} % Textpos units
\setlength{\TPVertModule}{\paperheight} % Textpos units
\usepackage{tikz}
\usetikzlibrary{overlay-beamer-styles}  % Overlay effects for TikZ
\usepackage{array}
\usepackage{diagbox}
\usepackage{booktabs}
\usepackage{svg}
\usepackage{xcolor,colortbl}


% Redefine `\rowcolor` to allow a beamer overlay specifier
% New syntax: \rewcolor<overlay>[color model]{color}[left overhang][right overhang]
\makeatletter
% Open `\noalign` and check for overlay specification:
\def\rowcolor{\noalign{\ifnum0=`}\fi\bmr@rowcolor}
\newcommand<>{\bmr@rowcolor}{%
	\alt#1%
	{\global\let\CT@do@color\CT@@do@color\@ifnextchar[\CT@rowa\CT@rowb}% Rest of original `\rowcolor`
	{\ifnum0=`{\fi}\@gooble@rowcolor}% End `\noalign` and gobble all arguments of `\rowcolor`.
}
% Gobble all normal arguments of `\rowcolor`:
\newcommand{\@gooble@rowcolor}[2][]{\@gooble@rowcolor@}
\newcommand{\@gooble@rowcolor@}[1][]{\@gooble@rowcolor@@}
\newcommand{\@gooble@rowcolor@@}[1][]{\ignorespaces}
\makeatother


\author{Odin Hoff Gardå}
\title{Q-læring}
\subtitle{INF100}


\begin{document}

\begin{frame}{Plan}

\vspace{4em}

\begin{enumerate}
	\setlength\itemsep{4em}
	\item \Large Kort introduksjon til Q-læring.
	\item \Large Workshop: Implementer Q-læring for å løse en labyrint.
\end{enumerate}

\end{frame}


\begin{frame}{Forsterkende Læring\footnote{Engelsk: Reinforcement Learning}}

\begin{columns}
	\begin{column}{0.5\textwidth}
		\begin{itemize}
			\setlength\itemsep{2.5em}
			\item Vi har en \textbf{agent} som \textbf{handler} i et \textbf{miljø}.
			\item Agenten lærer gjennom \textbf{belønning} basert på \textbf{tilstand} og \textbf{handling}.
			\item \textbf{Q-læring} er en form for forsterkende læring.
		\end{itemize}
	\end{column}
	\begin{column}{0.5\textwidth}  %%<--- here
		\begin{center}
\includesvg[width=\textwidth]{figs/Reinforcement_learning_diagram}
		\end{center}
	\end{column}
\end{columns}
\end{frame}

\begin{frame}{Q-læring: Oppsett}
	\begin{itemize}
		\setlength\itemsep{0.75em}
		\item Vi starter med:
		\begin{itemize}
			\setlength\itemsep{0.625em}
			\item En mengde $\mathcal{S}$ av mulige \textbf{tilstander}
			\item En mengde $\mathcal{A}$ av mulige \textbf{handlinger}
			\item Et par $(s, a)\in \mathcal{S}\times\mathcal{A}$ kalles et \textbf{tilstand-handlings-par}
			\item En \textbf{belønningsfunksjon} $R\colon\mathcal{S}\times\mathcal{A}\to\mathbb{R}$
		\end{itemize}
	\pause
	
	
	\item Ved å la agenten utforske miljøet ønsker vi å lære \textbf{Q-funksjonen} $$Q\colon\mathcal{S}\times\mathcal{A}\to\mathbb{R}$$ som gir oss en \textbf{Q-verdi} $Q(s,a)$ til hvert par $(s,a)\in\mathcal{S}\times\mathcal{A}$.
	
	\pause
	
	\item \textbf{Endelig mål:} For en $s\in\mathcal{S}$ så ønsker vi at $\arg\max_{a\in\mathcal{A}}Q(s,a)$ er den optimale handlingen for å maksimere forventet belønning.
	\end{itemize}
\end{frame}

\begin{frame}{Gjennomgående Eksempel: $3\times3$ Labyrint}
	%Enkelt eksempel fordi:
	%\begin{itemize}
	%	\item Diskret tid
	%	\item Statisk miljø (labyrinten endrer seg ikke)
	%	\item Få tilstander og handlinger (mengden $\mathcal{S}\times\mathcal{A}$ har få elementer)
	%\end{itemize}
	
	\begin{columns}
		\begin{column}{0.5\textwidth}
			\begin{center}
			\includesvg[width=0.6\textwidth]{figs/maze_1_with_coords.svg}
			\end{center}
		\end{column}
		\begin{column}{0.5\textwidth}%%<--- here
			\begin{table}[!hbt]
				\centering
				\begin{tabular}{m{1cm} m{2cm}}
				\includesvg[width=0.125\textwidth]{figs/open_tile} & Åpen rute\\
				\includesvg[width=0.125\textwidth]{figs/wall_tile} &  Vegg \\ 
				\includesvg[width=0.125\textwidth]{figs/agent_tile} & Agent\\
				\includesvg[width=0.125\textwidth]{figs/goal_tile} & Mål\\
				\end{tabular}
			\end{table}
		\end{column}
	\end{columns}
	
	\vspace{1em}
	
	\pause
	
	Mulige tilstander (agentens posisjon):
	$$\mathcal{S} =\lbrace (0,0), (1,0), (2,0), \textcolor{gray}{(0,1)}, (1,1), \textcolor{gray}{(2,1)}, (0,2), (1,2), \textcolor{gray}{(2,2)}\rbrace$$
	
	\pause
	
	Mulige handlinger (retninger å gå):
	$$\mathcal{A} = \{\text{venstre}, \text{høyre}, \text{opp}, \text{ned}\}$$
\end{frame}

\begin{frame}{Eksempel: Belønningsfunksjonen}
	La $s'$ være ruten i retning $a$ fra posisjon $s$.
	
	\vspace{1em}
	
	Definer belønningsfunksjonen $R\colon\mathcal{S}\times\mathcal{A}\to\mathbb{R}$ ved
	$$
	R(s,a) = \begin{cases}
		-1.0 & \text{hvis } s'\text{ er en veggrute},\\
		-0.1 & \text{hvis } s'\text{ er en åpen rute og }\\
		1.0 & \text{hvis } s'\text{ er målruten}.
	\end{cases}
	$$
	\begin{columns}
	\begin{column}{0.3\textwidth}
		\begin{center}
			\includesvg[width=\textwidth]{figs/maze_1_with_coords_no_agent.svg}
		\end{center}
	\end{column}
	\begin{column}{0.7\textwidth}%%<--- here
	\begin{itemize}
		\item<1-> \textcolor<3->{gray}{\textbf{Q:} Hva er $R((1,0), \text{høyre})$?}
		\item<2-> \textcolor<3->{gray}{\textbf{A:} $R((1,0), \text{høyre}) = 1.0$}
		\item<3-> \textcolor<5->{gray}{\textbf{Q:} Hva er $R((1,1), \text{venstre})$?}
		\item<4-> \textcolor<5->{gray}{\textbf{A:} $R((1,1), \text{venstre}) = -1.0$}
		\item<5->\textbf{Q:} Hva er $R((1,1), \text{opp})$?
		\item<6->\textbf{A:} $R((1,1), \text{opp}) = -0.1$
	\end{itemize}
	\end{column}
	\end{columns}
\end{frame}

\begin{frame}{Q-tabell}
	Hvis vi har endelig mange tilstander og handlinger, kan vi representere Q-funksjonen som en tabell (\textbf{Q-tabell}):
	\begin{columns}
	\begin{column}{0.5\textwidth}
	\begin{table}[!hbt]
	\centering
	\def\arraystretch{1.1}
	\setlength\tabcolsep{0.01\textwidth}
	\begin{tabular}{|c|c|c|c|c|}\hline
			\diagbox{$\textbf{s}$}{$\textbf{a}$} & $\textbf{venstre}$ & \only<1-2>{\cellcolor{orange!25}} $\textbf{høyre}$ & $\textbf{opp}$ & \only<3-4>{\cellcolor{orange!25}} $\textbf{ned}$\\ \hline
			\only<1-2>{\cellcolor{orange!25}}$(0,0)$ & 1.0 & \only<2>{\cellcolor{orange!75}}-0.5 & 0.1 & -0.3 \\ \hline
			$(1,0)$ & 0.2 & -0.3 & -1.0 & 0.6 \\ \hline
			$(2,0)$ & -0.4 & -0.1 & 0.3 & 0.7 \\ \hline
			$(0,1)$ & 0.4 & -0.9 & 1.0 & 0.2 \\ \hline
		 	\only<3-4>{\cellcolor{orange!25}}$(1,1)$ & 0.6 & -0.1 & 0.4 & \only<4>{\cellcolor{orange!75}}0.8 \\ \hline
			$(2,1)$ & 1.0 & 0.5 & 0.8 & -0.5 \\ \hline
			$(0,2)$ & -0.2 & 0.5 & -0.3 & -0.7 \\ \hline
			\rowcolor<5-6>{orange!25}$(1,2)$ & \only<6>{\cellcolor{orange!75}}0.7 & -1.0 & 0.1 & -0.5 \\ \hline
			\rowcolor<7-8>{orange!25}$(2,2)$ & 0.1 & 0.0 & \only<8>{\cellcolor{orange!75}}0.8 & 0.1 \\ \hline
	\end{tabular}
	\end{table}
	\end{column}
	\begin{column}{0.5\textwidth}%%<--- here
	
	\textbf{Spørsmål:}
	
	\begin{itemize}
	\item<1->\textcolor<3->{gray}{\textbf{Q:} Hva er $Q((0,0), \text{høyre})$?}
	\item<2->\textcolor<3->{gray}{\textbf{A:} $-0.5$}
	\item<3->\textcolor<5->{gray}{\textbf{Q:} Hva er $Q((1,1), \text{ned})$?}
	\item<4->\textcolor<5->{gray}{\textbf{A:} $0.8$}
	\item<5->\textcolor<7->{gray}{\textbf{Q:} Hva er $\max_{a\in\mathcal{A}}Q(s,a)$ når $s=(1,2)$?}
	\item<6->\textcolor<7->{gray}{\textbf{A:} $0.7$}
	\item<7->\textbf{Q:} Hva er $\max_{a\in\mathcal{A}}Q(s,a)$ når $s=(2,2)$?
	\item<8->\textbf{A:} $0.8$
	\end{itemize}
	\end{column}
	\end{columns}
\end{frame}

\begin{frame}{Eksempel: Handling basert på Q-tabell}
	La $\pi^*\colon\mathcal{S}\to\mathcal{A}$ være funksjonen gitt ved $\pi^*(s) = \arg\max_{a\in\mathcal{A}}Q(s,a)$.
	\begin{columns}
		\begin{column}{0.5\textwidth}
			\begin{table}[!hbt]
				\centering
				\def\arraystretch{1.1}
				\setlength\tabcolsep{0.01\textwidth}
				\begin{tabular}{|c|c|c|c|c|}\hline
					\diagbox{$\textbf{s}$}{$\textbf{a}$} & \only<4>{\cellcolor{orange!75}} $\textbf{venstre}$ & \only<6>{\cellcolor{orange!75}} $\textbf{høyre}$ & \only<2>{\cellcolor{orange!75}} $\textbf{opp}$ & $\textbf{ned}$\\ \hline
					$(0,0)$ & 1.0 & -0.5 & 0.1 & -0.3 \\ \hline
					$(1,0)$ & 0.2 & -0.3 & -1.0 & 0.6 \\ \hline
					$(2,0)$ & -0.4 & -0.1 & 0.3 & 0.7 \\ \hline
					\rowcolor<1-2>{orange!35}$(0,1)$ & 0.4 & -0.9 & \only<2>{\cellcolor{orange!75}}1.0 & 0.2 \\ \hline
					$(1,1)$ & 0.6 & -0.1 & 0.4 & 0.8 \\ \hline
					\rowcolor<3-4>{orange!35}$(2,1)$ & \only<4>{\cellcolor{orange!75}}1.0 & 0.5 & 0.8 & -0.5 \\ \hline
					\rowcolor<5-6>{orange!35}$(0,2)$ & -0.2 & \only<6>{\cellcolor{orange!75}}0.5 & -0.3 & -0.7 \\ \hline
					$(1,2)$ & 0.7 & -1.0 & 0.1 & -0.5 \\ \hline
					$(2,2)$ & 0.1 & 0.0 & 0.8 & 0.1 \\ \hline
				\end{tabular}
			\end{table}
		\end{column}
		\begin{column}{0.5\textwidth}%%<--- here
			
			\textbf{Spørsmål:}
					
			\begin{itemize}
			\item<1->\textcolor<3->{gray}{\textbf{Q:} Hva er $\pi^*((0,1))$?}
			\item<2->\textcolor<3->{gray}{\textbf{A:} $\text{opp}$}
			\item<3->\textcolor<5->{gray}{\textbf{Q:} Hva er $\pi^*((2,1))$?}
			\item<4->\textcolor<5->{gray}{\textbf{A:} $\text{venstre}$}
			\item<5->\textbf{Q:} Hva er $\pi^*((0,2))$?
			\item<6->\textbf{A:} $\text{høyre}$
			%\item\textbf{Q:} Hva er $\pi^*((1,1))$?
			%\item\textbf{A:} $\text{ned}$
			\end{itemize}	
		\end{column}
	\end{columns}
\end{frame}

\hidelogo
\begin{frame}{Hvordan lære Q-funksjonen?}
	Vi starter med $Q(s,a)=0$ for alle par $(s,a)\in\mathcal{S}\times\mathcal{A}$. (En Q-tabell hvor alle verdiene er $0$.)
	
	\vspace{1em}
	
	\pause
	
	Vi har to \textbf{læringsparametere} (begge tall mellom $0$ og $1$):
	\begin{itemize}
		\item $\alpha$: \textbf{læringsrate} (learning rate) og
		\item $\gamma$: \textbf{rabattfaktor} (discount factor).
	\end{itemize}
	
	\pause
	
	\vspace{1em}
	\textbf{Q-læringsalgoritmen (én episode):}
	
	\begin{enumerate}
		\item Agenten er i posisjon $s_t$ ved tid $t$. Velg en handling $a_t$. Ved å utføre $a_t$ i $s_t$ treffer vi $s_{t+1}$.
		
		\pause
		
		\item Vi oppdaterer $Q(s_t,a_t)$ med følgende regel:
		$$
		Q(s_t, a_t)\leftarrow(1-\alpha)Q(s_t, a_t)+\alpha\left(R(s_t, a_t)+\gamma\max_{a\in\mathcal{A}} Q(s_{t+1}, a)\right).
		$$
		
		\pause
		
		\item Gjenta fra steg 1 med $s_{t+1}$ (stopp hvis $s_{t+1}$ er en terminaltilstand).
	\end{enumerate}

\end{frame}

\begin{frame}{Oppdatering av Q-funksjonen}
	$$
	(1-\alpha)\underbracket{Q(s_t, a_t)}_{\text{nåværende Q-verdi}}+\,\alpha\left(\overbracket{R(s_t, a_t)}^{\substack{\text{Umiddelbar}\\\text{ belønning}}}+\gamma\underbracket{\max_{a\in\mathcal{A}} Q(s_{t+1}, a)}_{\substack{\text{estimert høyeste}\\ \text{fremtidig belønning}}}\right).
	$$
	\begin{itemize}
		\setlength\itemsep{2em}
		\item Gammel og ny kunnskap kombineres ($\alpha$ bestemmer balansen mellom de).
		\item Belønningen for å utføre $a_t$ i tilstand $s_t$ påvirker den nye Q-verdien.
		\item Hvor mye vi bryr oss om fremtiden bestemmes av $\gamma$.
		%\item Summanden $\gamma\max_{a\in\mathcal{A}} Q(s_{t+1}, a)$ tillater flyt av informasjon mellom forskjellige tilstander.
	\end{itemize}
\end{frame}

\begin{frame}{$\epsilon$-grådig Q-læring: Hvordan velge $a_t$?}
	\vspace{1em}
	La $\epsilon$ være et tall mellom $0$ og $1$. I tilstand $s_t$, velg $a_t$ på følgende måte:
	
	\vspace{1em}
	
	\begin{enumerate}
		\setlength\itemsep{2em}
		\item Med sannsynlighet $\epsilon$, velg $a_t$ tilfeldig.
		\item Med sannsynlighet $1-\epsilon$, velg $a_t=\pi^*(s_t)=\arg\max_{a\in\mathcal{A}}Q(s_t,a)$.
	\end{enumerate}
	
	\vspace{1em}
	
	
	Vi reduserer vanligvis verdien av $\epsilon$ gjennom læringen slik at agenten utforsker mest i starten men gradvis baserer valgene på lært kunnskap.
	
\end{frame}


\begin{frame}{Eksempel: Læringsteg 1}
	Sett $\alpha=0.8$ og $\gamma=0.5$. La $s_t=(1,2)$ og $a_t=\text{opp}$.
	\vspace{-1em}
	\begin{columns}
	\begin{column}{0.4\textwidth}
	\begin{center}
	\includesvg[width=0.8\textwidth]{figs/maze_1_with_coords_go_up.svg}
	\end{center}
	\end{column}
	\begin{column}{0.6\textwidth}%%<--- here
	\begin{table}[!hbt]
		\centering
		\def\arraystretch{1.0}
		\setlength\tabcolsep{0.01\textwidth}
		\begin{tabular}{|c|c|c|c|c|}\hline
			\diagbox{$\textbf{s}$}{$\textbf{a}$} & $\textbf{venstre}$ & $\textbf{høyre}$ & $\textbf{opp}$ & $\textbf{ned}$\\ \hline
			\multicolumn{5}{|c|}{\vdots} \\ \hline
			$(1,0)$ & 0.1 & 1.0 & -0.8 & 0.2 \\ \hline
			$(1,1)$ & -0.6 & -0.8 & 0.9 & 0.2 \\ \hline
			\rowcolor{orange!35}$(1,2)$ & -0.4 & -1.0 & \cellcolor{orange!70}0.3 & -0.9 \\ \hline
			\multicolumn{5}{|c|}{\vdots}\\ \hline
		\end{tabular}
	\end{table}
	\end{column}
	\end{columns}		
	$$
	Q(s_t, a_t)\leftarrow\underbracket{(1-\alpha)}_{0.2}\underbracket{Q(s_t, a_t)}_{0.3}+\underbracket{\alpha}_{0.8}\left(\underbracket{R(s_t, a_t)}_{-0.1}+\underbracket{\gamma}_{0.5}\underbracket{\max_{a\in\mathcal{A}} Q(\overbracket{s_{t+1}}^{(1,1)}, a)}_{0.9}\right).
	$$
	Ny Q-verdi: $Q((1,2), \text{opp})=0.2\cdot0.3+0.8(-0.1+0.5\cdot0.9)=\textbf{0.34}$
\end{frame}

\begin{frame}{Eksempel: Læringsteg 2}
	Nå er $s_t=(1,1)$. La $a_t=\text{venstre}$.
	\vspace{-1em}
	\begin{columns}
		\begin{column}{0.4\textwidth}
			\begin{center}
				\includesvg[width=0.8\textwidth]{figs/maze_2_with_coords_go_left.svg}
			\end{center}
		\end{column}
		\begin{column}{0.6\textwidth}%%<--- here
			\begin{table}[!hbt]
				\centering
				\def\arraystretch{1.0}
				\setlength\tabcolsep{0.01\textwidth}
				\begin{tabular}{|c|c|c|c|c|}\hline
					\diagbox{$\textbf{s}$}{$\textbf{a}$} & $\textbf{venstre}$ & $\textbf{høyre}$ & $\textbf{opp}$ & $\textbf{ned}$\\ \hline
					\multicolumn{5}{|c|}{\vdots} \\ \hline
					$(1,0)$ & 0.1 & 1.0 & -0.8 & 0.2 \\ \hline
					\rowcolor{orange!35}$(1,1)$ & \cellcolor{orange!70}-0.6 & -0.8 & 0.9 & 0.2 \\ \hline
					$(1,2)$ & -0.4 & -1.0 & \textit{0.34} & -0.9 \\ \hline
					\multicolumn{5}{|c|}{\vdots}\\ \hline
				\end{tabular}
			\end{table}
		\end{column}
	\end{columns}		
	$$
	Q(s_t, a_t)\leftarrow\underbracket{(1-\alpha)}_{0.2}\underbracket{Q(s_t, a_t)}_{-0.6}+\underbracket{\alpha}_{0.8}\left(\underbracket{R(s_t, a_t)}_{-1.0}+\underbracket{\gamma}_{0.5}\underbracket{\max_{a\in\mathcal{A}} Q(\overbracket{s_{t+1}}^{(1,1)}, a)}_{0.9}\right).
	$$
	Ny Q-verdi: $Q((1,1), \text{venstre})=\textbf{-0.56}$
\end{frame}

\begin{frame}{Eksempel: Læringsteg 3}
	Nå er $s_t=(1,1)$. La $a_t:=\pi^*(s_t)=\text{opp}$.
	\vspace{-1em}
	\begin{columns}
		\begin{column}{0.4\textwidth}
			\begin{center}
				\includesvg[width=0.8\textwidth]{figs/maze_2_with_coords_go_up.svg}
			\end{center}
		\end{column}
		\begin{column}{0.6\textwidth}%%<--- here
			\begin{table}[!hbt]
				\centering
				\def\arraystretch{1.0}
				\setlength\tabcolsep{0.01\textwidth}
				\begin{tabular}{|c|c|c|c|c|}\hline
					\diagbox{$\textbf{s}$}{$\textbf{a}$} & $\textbf{venstre}$ & $\textbf{høyre}$ & $\textbf{opp}$ & $\textbf{ned}$\\ \hline
					\multicolumn{5}{|c|}{\vdots} \\ \hline
					$(1,0)$ & 0.1 & 1.0 & -0.8 & 0.2 \\ \hline
					\rowcolor{orange!35}$(1,1)$ & \textit{-0.56} & -0.8 & \cellcolor{orange!70}0.9 & 0.2 \\ \hline
					$(1,2)$ & -0.4 & -1.0 & \textit{0.34} & -0.9 \\ \hline
					\multicolumn{5}{|c|}{\vdots}\\ \hline
				\end{tabular}
			\end{table}
		\end{column}
	\end{columns}		
	$$
	Q(s_t, a_t)\leftarrow\underbracket{(1-\alpha)}_{0.2}\underbracket{Q(s_t, a_t)}_{0.9}+\underbracket{\alpha}_{0.8}\left(\underbracket{R(s_t, a_t)}_{-0.1}+\underbracket{\gamma}_{0.5}\underbracket{\max_{a\in\mathcal{A}} Q(\overbracket{s_{t+1}}^{(1,0)}, a)}_{1.0}\right).
	$$
	Ny Q-verdi: $Q((1,1), \text{opp})=\textbf{0.5}$
\end{frame}

\begin{frame}{Eksempel: Læringsteg 4}
	Nå er $s_t=(1,0)$. La $a_t:=\pi^*(s_t)=\text{høyre}$.
	\vspace{-1em}
	\begin{columns}
		\begin{column}{0.4\textwidth}
			\begin{center}
				\includesvg[width=0.8\textwidth]{figs/maze_3_with_coords_go_right.svg}
			\end{center}
		\end{column}
		\begin{column}{0.6\textwidth}%%<--- here
			\begin{table}[!hbt]
				\centering
				\def\arraystretch{1.0}
				\setlength\tabcolsep{0.01\textwidth}
				\begin{tabular}{|c|c|c|c|c|}\hline
					\diagbox{$\textbf{s}$}{$\textbf{a}$} & $\textbf{venstre}$ & $\textbf{høyre}$ & $\textbf{opp}$ & $\textbf{ned}$\\ \hline
					\multicolumn{5}{|c|}{\vdots} \\ \hline
					\rowcolor{orange!35}$(1,0)$ & 0.1 & \cellcolor{orange!70}1.0 & -0.8 & 0.2 \\ \hline
					$(1,1)$ & \textit{-0.56} & -0.8 & \textit{0.5} & 0.2 \\ \hline
					$(1,2)$ & -0.4 & -1.0 & \textit{0.34} & -0.9 \\ \hline
					\multicolumn{5}{|c|}{\vdots}\\ \hline
				\end{tabular}
			\end{table}
		\end{column}
	\end{columns}		
	$$
	Q(s_t, a_t)\leftarrow\underbracket{(1-\alpha)}_{0.2}\underbracket{Q(s_t, a_t)}_{1.0}+\underbracket{\alpha}_{0.8}\left(\underbracket{R(s_t, a_t)}_{1.0}+\underbracket{\gamma}_{0.5}\underbracket{\max_{a\in\mathcal{A}} Q(\overbracket{s_{t+1}}^{(2,0)}, a)}_{0.0}\right).
	$$
	Ny Q-verdi: $Q((1,0), \text{høyre})=\textbf{1.0}$
\end{frame}

\begin{frame}{Workshop}

\vspace{1em}

Nå er det din tur til å implementere Q-læring!

\vspace{1em}

\begin{itemize}
	\setlength\itemsep{1.1em}
	\item Gå til {\large\url{https://github.com/odinhg/Q-Learning-Tutorial}} (eller skann QR-koden).
		
	\begin{center}
		\includesvg[width=0.25\textwidth]{figs/qr-code.svg}
	\end{center}

	\item Spør en gruppeleder eller meg dersom du har spørsmål.
	
\end{itemize}




\end{frame}



\end{document}